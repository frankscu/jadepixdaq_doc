\documentclass[12pt,a4paper]{article}

\usepackage{amsmath}
\usepackage{listings}
\usepackage{xcolor}
\usepackage{graphicx}
\usepackage{geometry}
\geometry{left=2.5cm,right=2.5cm,top=2.5cm,bottom=2.5cm}

\lstdefinelanguage{Json}{
    basicstyle=\normalfont\ttfamily,
    numbers=left,
    numberstyle=\scriptsize,
    stepnumber=1,
    numbersep=8pt,
    showstringspaces=false,
    breaklines=true,
    frame=lines,
    backgroundcolor=\color{background},
    literate=
     *{0}{{{\color{numb}0}}}{1}
      {1}{{{\color{numb}1}}}{1}
      {2}{{{\color{numb}2}}}{1}
      {3}{{{\color{numb}3}}}{1}
      {4}{{{\color{numb}4}}}{1}
      {5}{{{\color{numb}5}}}{1}
      {6}{{{\color{numb}6}}}{1}
      {7}{{{\color{numb}7}}}{1}
      {8}{{{\color{numb}8}}}{1}
      {9}{{{\color{numb}9}}}{1}
      {:}{{{\color{punct}{:}}}}{1}
      {,}{{{\color{punct}{,}}}}{1}
      {\{}{{{\color{delim}{\{}}}}{1}
      {\}}{{{\color{delim}{\}}}}}{1}
      {[}{{{\color{delim}{[}}}}{1}
      {]}{{{\color{delim}{]}}}}{1},
}

\begin{document}

\title{JadePix DAQ user guide}

\author{IHEP, CAS, China \\
\small{Version 0,1}
}

\tableofcontents

\section{Compiling and installation}
\label{compiling-and-installation}

\emph{cmake} will configure the installation and prepare the makefiles.
It searches for all the required files. As a standard, it is executed in
the \emph{build} folder.

\subsection{Quick installation for UNIX/LINUX}

\begin{lstlisting}[language=Bash]
git clone https://github.com/cepc/kc705.git
mkdir -p build
cmake ..
make install -j N
\end{lstlisting}

\subsection{Use for radioactive experiments}
\label{use-for-radioactive-experiments}

\begin{lstlisting}[language=Bash]
cd kc705 
git checkout v1.x
cd build
cmake ..
make install -j N
\end{lstlisting}

\subsection{Use for beam tests}
\label{use-for-beam-tests}

\begin{lstlisting}[language=Bash]
cd kc705 
git checkout master
cd build
cmake ..
make install -j N
\end{lstlisting}

After installation, the folder \emph{bin} and \emph{lib} will be installed in
kc705 directory. In directory of bin, there are the following exe programs.

\begin{itemize}
	\item
	      \emph{SystemTest} are used for data acquring with json configuration.
	\item
	      \emph{ManagterTest} are used for data acquring without json configuration.
	\item
	      \emph{OptionTest} are used for json debug.
	\item
	      \emph{ReaderTest} are used for read test of DAQ.
	\item
		\emph{dd.exe}, \emph{memread.exe}, \emph{streamread.exe}, \emph{memwrite.exe} 
		and streamwrite.exe are coppied from Xilinx company, which are used for firmware 
		debug.
\end{itemize}

\subsection{Advaned installation for UNIX/LINUX}
\label{advaned-installation-for-unixlinux}

cmake has several options (\texttt{cmake\ -D\ OPTION=ON/OFF}) to
activate or deactivate programs which will be built. For advanced use,
cmake-gui/ccmake will help to make selection visiable.

The following prints the options.

\begin{itemize}
	\item
	      JADE\_BUILD\_CORE\_PYBIND
	\item
	      JADE\_BUILD\_LEGACY\_CORE\_PYTHON
	\item
	      JADE\_BUILD\_GUI
\end{itemize}

\section{Run examples}
\label{run-examples}

\subsection{Run and configure}
\label{run-and-configure}

User input command SystemTest\ -c\ kc705/software/conf/sample.json will start data
taking. The json file is the configuration of the JadePixDAQ.

\begin{lstlisting}[language=Json]
	{
		"System":"JadeCore",
		"JadeCore":{
		  "JadeManager":{
		    "type":"default",
		    "parameter":{
			"SecPerLoop": 10,
			"N_Loops": 3,
			"ChipAddress": "CHIPA1"
		    }
		  },
		  "JadeRead":{
		    "type":"default",
		    "parameter":{
			"PATH":"//./xillybus_read_32"
		    }
		  },
		  "JadeWrite":{
		    "type":"default",
		    "parameter":{
			"PATH":"F:\\data\\output",
			"DISABLE_FILE_WRITE":false
		    }
		  },
		  "JadeFilter":{
		    "type":"default",
		    "parameter":{
			"something": "somthing"
		    }
		  },
		  "JadeMonitor":{
		    "type":"default",
		    "parameter":{
			"PRINT_EVENT_N":40000,
			"PRINT_EVENT_DISCONTINUOUS":true
		    }
		  },
		  "JadeRegCtrl":{
		    "type":"default",
		    "parameter":{
			"PATH":"//./xillybus_mem_8",
			"command_list":{
			  "RESET":{
			    "address":1,
			    "default_val":15
			  },
			  "SET":{
			    "address":2,
			    "default_val":10
			  },
			  "START":{
			    "address":3,
			    "default_val":15
			  },
			  "STOP":{
			    "address":4,
			    "default_val":15
			  },
			  "CHIPA1":{
			    "address":8,
			    "default_val":0
			  },
			  "CHIPA2":{
			    "address":8,
			    "default_val":1
			  },
			  "CHIPA3":{
			    "address":8,
			    "default_val":2
			  },
			  "CHIPA4":{
			    "address":8,
			    "default_val":3
			  },
			  "CHIPA5":{
			    "address":8,
			    "default_val":4
			  },
			  "CHIPA6":{
			    "address":8,
			    "default_val":5
			  },
			  "CHIPA7":{
			    "address":8,
			    "default_val":6
			  },
			  "CHIPA8":{
			    "address":8,
			    "default_val":7
			  },
			  "CHIPA9":{
			    "address":8,
			    "default_val":8
			  },
			  "CHIPA10":{
			    "address":8,
			    "default_val":9
			  }
			},
			"status_list":{
			  "FIFO_MODE":{
			    "address":16,
			    "value_list":{
				"NOT_FULL":0,
				"FULL":1
			    }
			  },
			  "RUN_MODE":{
			    "address":16,
			    "value_list":{
				"STOPPED":0,
				"STARTED":1
			    }
			  }
			}
		    }
		  }
		}
	    }
\end{lstlisting}

As the above shows, the JadeCore is the system, which has servaral
sections. 
\begin{itemize}
\item{
	\emph{JadeRead}: the PATH means the device in PC of kc705 FPGA
board. In Linux, it is /dev/xillybus\_read\_32}
\item{JadeWrite: te PATH
means the output raw data file. User has no worry about overwriting the
file, since the output file will has time suffix.} 
\item{
	\emph{JadeMonitor}: the
PRINT\_EVENT\_N is the number to print frames.
PRINT\_EVENT\_DISCONTINUOUS is the monitor of data continuous. It should
be true in radioactive test, which false in beam tests.}
\item{
	\emph{JadeRegCtrl}:
the PATH means the device in PC of kc705 to confugre register. it is
/dev/xillybus\_mem\_8 in Linux. Other parameters are the command lists
of register. For more information, please browse the link
https://github.com/cepc/kc705/blob/v1.2/doc/Firmware\_v1.2\_Info.md.}


\end{document}
