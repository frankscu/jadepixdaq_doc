\documentclass[12pt,a4paper]{article}

\usepackage{graphicx}
\usepackage[colorlinks,linkcolor=blue]{hyperref}
\usepackage{listings}
\usepackage{fontspec}
\usepackage{enumerate}
\usepackage{xcolor}
\usepackage{color}
\lstset{basicstyle=\footnotesize\ttfamily, numbers=left,
	keywordstyle=\color{blue}, numberstyle={\tiny\color{black}}, numbersep=5pt,
	commentstyle=\small\color{red}, backgroundcolor=\color{white},
	showspaces=false, showtabs=false, frame=single, framexleftmargin=5mm,
	rulesepcolor=\color{red!20!green!20!blue!20}, tabsize=4, breaklines=true,
	extendedchars=false}

\title{JadePix DAQ user guide}

\author{IHEP, CAS, China \\
	\small{Version 0,1}
}

\begin{document}

\maketitle

\newpage

\tableofcontents

\newpage

\section{Compiling and installation}
\label{compiling-and-installation}

\emph{cmake} will configure the installation and prepare the makefiles.
It searches for all the required files. As a standard, it is executed in
the \emph{build} folder.

\subsection{Quick installation for UNIX/LINUX}

\begin{lstlisting}[language=bash]
git clone https://github.com/cepc/kc705.git
mkdir -p build
cmake ..
make install -j N
\end{lstlisting}

\subsection{Use for radioactive experiments}
\label{use-for-radioactive-experiments}

\begin{lstlisting}[language=bash]
cd kc705 
git checkout v1.x
cd build
cmake ..
make install -j N
\end{lstlisting}

\subsection{Use for beam tests}
\label{use-for-beam-tests}

\begin{lstlisting}[language=Bash]
cd kc705 
git checkout master
cd build
cmake ..
make install -j N
\end{lstlisting}

After installation, the folder \emph{bin} and \emph{lib} will be installed in
kc705 directory. In directory of bin, there are the following exe programs.

\begin{enumerate}
	\item{\emph{SystemTest} are used for data acquring with json configuration.}
	\item{\emph{ManagterTest} are used for data acquring without json configuration.}
	\item{\emph{OptionTest} are used for json debug.}
	\item{\emph{ReaderTest} are used for read test of DAQ.}
	\item{\emph{dd.exe}, \emph{memread.exe}, \emph{streamread.exe}, \emph{memwrite.exe}
	      and \emph{streamwrite.exe} are coppied from Xilinx company for firmware debug.}
\end{enumerate}

\subsection{Advaned installation for UNIX/LINUX}
\label{advaned-installation-for-unixlinux}

cmake has several options (\texttt{cmake\ -D\ OPTION=ON/OFF}) to activate or
deactivate programs which will be built. For advanced use, cmake-gui/ccmake will
help to make selection visiable.

The following prints the options.

\begin{enumerate}
	\item{JADE\_BUILD\_CORE\_PYBIND}
	\item{JADE\_BUILD\_LEGACY\_CORE\_PYTHON}
	\item{JADE\_BUILD\_GUI}
\end{enumerate}

\section{Run examples}
\label{run-examples}

\subsection{Run and configure}
\label{run-and-configure}

User input the following command will start data taking

\begin{lstlisting}[language=bash]
SystemTest -c kc705/software/conf/sample.json 
\end{lstlisting}

data taking. The json file is the configuration of the JadePixDAQ.

\begin{lstlisting}
	{
		"System":"JadeCore",
		"JadeCore":{
		  "JadeManager":{
		    "type":"default",
		    "parameter":{
			"SecPerLoop": 10,
			"N_Loops": 3,
			"ChipAddress": "CHIPA1"
		    }
		  },
		  "JadeRead":{
		    "type":"default",
		    "parameter":{
			"PATH":"//./xillybus_read_32"
		    }
		  },
		  "JadeWrite":{
		    "type":"default",
		    "parameter":{
			"PATH":"F:\\data\\output",
			"DISABLE_FILE_WRITE":false
		    }
		  },
		  "JadeFilter":{
		    "type":"default",
		    "parameter":{
			"something": "somthing"
		    }
		  },
		  "JadeMonitor":{
		    "type":"default",
		    "parameter":{
			"PRINT_EVENT_N":40000,
			"PRINT_EVENT_DISCONTINUOUS":true
		    }
		  },
		  "JadeRegCtrl":{
		    "type":"default",
		    "parameter":{
			"PATH":"//./xillybus_mem_8",
			"command_list":{
			  "RESET":{
			    "address":1,
			    "default_val":15
			  },
			  "SET":{
			    "address":2,
			    "default_val":10
			  },
			  "START":{
			    "address":3,
			    "default_val":15
			  },
			  "STOP":{
			    "address":4,
			    "default_val":15
			  },
			  "CHIPA1":{
			    "address":8,
			    "default_val":0
			  },
			  "CHIPA2":{
			    "address":8,
			    "default_val":1
			  },
			  "CHIPA3":{
			    "address":8,
			    "default_val":2
			  },
			  "CHIPA4":{
			    "address":8,
			    "default_val":3
			  },
			  "CHIPA5":{
			    "address":8,
			    "default_val":4
			  },
			  "CHIPA6":{
			    "address":8,
			    "default_val":5
			  },
			  "CHIPA7":{
			    "address":8,
			    "default_val":6
			  },
			  "CHIPA8":{
			    "address":8,
			    "default_val":7
			  },
			  "CHIPA9":{
			    "address":8,
			    "default_val":8
			  },
			  "CHIPA10":{
			    "address":8,
			    "default_val":9
			  }
			},
			"status_list":{
			  "FIFO_MODE":{
			    "address":16,
			    "value_list":{
				"NOT_FULL":0,
				"FULL":1
			    }
			  },
			  "RUN_MODE":{
			    "address":16,
			    "value_list":{
				"STOPPED":0,
				"STARTED":1
			    }
			  }
			}
		    }
		  }
		}
			}
\end{lstlisting}

As the above shows, the JadeCore is the system, which has servaral
sections.
\begin{enumerate}
	\item{\textbf{JadeRead}: the PATH means the device in PC of kc705 FPGA board.
	      In Linux, it is /dev/xillybus\_read\_32}
	\item{\textbf{JadeWrite}: the PATH means the output raw data file. User has no
	      worry about overwriting the file, since the output file will has time
	suffix.}
	\item{\textbf{JadeMonitor}: the PRINT\_EVENT\_N is the number of print frames.
	      PRINT\_EVEN\_DISCONTINUOUS is the monitor of data continuous. It should
	be true in radioactive test, which false in beam tests.}
	\item{\textbf{JadeRegCtrl}: the PATH means the device in PC of kc705 to
	      confugre registers. It is /dev/xillybus\_mem\_8 in Linux. Other
	      parameters are the command lists of register. For more information,
	      please browse the link
	      \href{https://github.com/cepc/kc705/blob/v1.2/doc/Firmware\_v1.2\_Info.md}{firmware
	mannaul}}
\end{enumerate}


\end{document}
